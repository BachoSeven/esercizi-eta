\documentclass[a4paper]{article}

\usepackage[T1]{fontenc}
% many useful symbols
\usepackage{textcomp}
\usepackage[italian]{babel}
\usepackage{hyperref}
\usepackage{amsmath, amssymb, amsthm}
\usepackage{mathtools}
% for \lightning
\usepackage{stmaryrd}
\usepackage{geometry}
\usepackage{tikz-cd}
\usepackage{cancel}

% Remove indentation globally
\setlength{\parindent}{0pt}
% Have blank lines between paragraphs
\usepackage[parfill]{parskip}

\hypersetup{
    colorlinks = true, % links instead of boxes
    urlcolor   = cyan, % external hyperlinks
    linkcolor  = blue, % internal links
    citecolor  = cyan   % citations
}

\newcommand{\R}{\mathbb{R}}
\newcommand{\C}{\mathbb{C}}
\newcommand{\Q}{\mathbb{Q}}
\newcommand{\N}{\mathbb{N}}
\newcommand{\A}{\mathbb{A}}
\newcommand{\Z}{\mathbb{Z}}

% use bullets for items
\renewcommand{\labelitemii}{$\circ$}

\newcommand{\im}{\operatorname{im}}
\newcommand{\coker}{\operatorname{coker}}

\newcommand\numberthis{\addtocounter{equation}{1}\tag{\theequation}}

% Display math
\newcommand{\ssfrac}[2]{
	\raisebox{+0.3ex}{$#1$}
	/
	\raisebox{-0.3ex}{$#2$}
}
% Inline math
\newcommand{\sfrac}[2]{
	\raisebox{+0.3ex}{\scalebox{0.9}{$#1$}}
	/
	\raisebox{-0.3ex}{\scalebox{0.9}{$#2$}}
}

 % No white line in equal arrows
\usetikzlibrary{decorations.markings}
\tikzset{double line with arrow/.style args={#1,#2}{decorate,decoration={markings,%
                    mark=at position 0 with {\coordinate (ta-base-1) at (0,1pt);
                            \coordinate (ta-base-2) at (0,-1pt);},
                    mark=at position 1 with {\draw[#1] (ta-base-1) -- (0,1pt);
                            \draw[#2] (ta-base-2) -- (0,-1pt);
                        }}}}
\tikzset{Equal/.style={-,double line with arrow={-,-}}}

% Arrow
\newcommand\Iso{\xrightarrow{
                \,\smash{\raisebox{-0.65ex}{\ensuremath{\scriptstyle\sim}}}\,}}


\newtheorem{theorem}{Theorem}[section]
\newtheorem{lemma}[theorem]{Lemma}

\theoremstyle{definition}
\newtheorem{definition}[theorem]{Definition}

\theoremstyle{definition}
\newtheorem{example}[theorem]{Example}

\theoremstyle{remark}
\newtheorem*{remark}{Remark}

\theoremstyle{definition}
\newtheorem{exercise}{Esercizio}[section]
\newtheorem*{exercise*}{Esercizio}

\title{Elementi di Topologia Algebrica - Gruppo 4\vspace{-1cm}}
\author{}
\begin{document}

\maketitle

\textbf{Esercizio 1}

\textbf{Esercizio 2}

\textbf{Esercizio 3}
Vogliamo dimostrare l'esattezza della succcesione
\[
    [C_f, Z]^0 \overset{i_1^*}{\to} [Y, Z]^0 \overset{f^*}{\to} [X, Z]^0
\]
nel termine centrale. D'altronde, dato un rappresentante $\varphi$ di una classe in $[Y, Z]^0$, abbiamo che $f^*([\varphi]) = [\varphi\circ f]$ è banale se e solo se $\varphi \circ f$ è omotopicamente banale, ma per l'esercizio 2 questo succede
esattamente quando $\varphi$ fattorizza attraverso $C_f$, ovvero quando proviene tramite $i_1^*$ da un elemento di $[C_f, Z]^0$. Abbiamo quindi dimostrato che $\ker f^* = \im i_1^*$.

Per mostrare che le due successioni sono equivalenti nella categoria $\mathbf{hTop}$ degli spazi topologici con morfismi dati dalle classi di omotopia di mappa continue, mostriamo innanzitutto che, nella seguente situazione, esiste una equivalenza omotopica fra $C_{i_1}$ e $SX$:
\begin{align}\label{eq:1}
    X\overset{f}{\to} Y\overset{i_1}{\to} C_f \to C_{i_1}
\end{align}
Fatto ciò, l'equivalenza omotopica fra i termini successivi delle due successioni segue osservando che ogni tripla di mappa successiva nella prima successione è della forma \eqref{eq:1}.

È sufficiente mostrare che l'inclusione di $CY$ in $C_{i_1}$ è una cofibrazione: infatti, essendo il cono $CY$ contraibile ne segue che la proiezione al quoziente $C_{i_1} \to C_{i_1}/CY$ è una equivalenza omotopica, e quest'ultimo termine è evidentemente omeomrfo alla sospensione $SX$.

D'altronde, abbiamo visto a lezione che le cofibrazioni sono stabili per pushout. Consideriamo quindi il seguente diagramma:
\[
    \begin{tikzcd}
        X \arrow[r,"i"] \arrow[d,"f"] & CX \arrow[d] \\
        Y \arrow[r,"i_1"] \arrow[d] & C_f \arrow[d] \\
        CY \arrow[r] & C_{i_1}
    \end{tikzcd}
\]
Visto l'inclusione $i$ di $X$ nel suo cono è una cofibrazione e $C_f$ è il pushout del quadrato superiore, dunque $i_1$ è una cofibrazione, ma allora visto che $C_{i_1}$ è il pushout del quadrato inferiore otteniamo $C_f \to C_{i_1}$ è una cofibrazione, come voluto.

Ci resta da stabilire che il diagramma dato dalle due successioni collegate verticalmente dalle equivalenze costruite è commutativo in $\mathbf{hTop}$. Per dirlo, senza perdita di generalità basta considerare il diagramma
\[
    \begin{tikzcd}
        C_{i_1} \arrow[r,"i_3"] \arrow[d] & C_{i_2} \arrow[d] \\
        SX \arrow[r,"Sf"] & SY
    \end{tikzcd}
\]
ed esibire un'omotopia fra i due percorsi:
\begin{align*}
    H & : C_{i_1} \times I \to SY                            \\
    H & ([x,t], s)  = \begin{cases}
                          [f (x), st]  & \text{se } [x,t]\in C_f \\
                          [x,s+(1-s)t] & \text{se } [x,t]\in CY
                      \end{cases}
\end{align*}


\end{document}
