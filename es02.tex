\documentclass[a4paper]{article}

\usepackage[T1]{fontenc}
% many useful symbols
\usepackage{textcomp}
\usepackage[italian]{babel}
\usepackage{hyperref}
\usepackage{amsmath, amssymb, amsthm}
\usepackage{mathtools}
% for \lightning
\usepackage{stmaryrd}
\usepackage{geometry}
\usepackage{tikz-cd}
\usepackage{cancel}

% Remove indentation globally
\setlength{\parindent}{0pt}
% Have blank lines between paragraphs
\usepackage[parfill]{parskip}

\hypersetup{
    colorlinks = true, % links instead of boxes
    urlcolor   = cyan, % external hyperlinks
    linkcolor  = blue, % internal links
    citecolor  = cyan   % citations
}

\newcommand{\R}{\mathbb{R}}
\newcommand{\C}{\mathbb{C}}
\newcommand{\Q}{\mathbb{Q}}
\newcommand{\N}{\mathbb{N}}
\newcommand{\A}{\mathbb{A}}
\newcommand{\Z}{\mathbb{Z}}

% use bullets for items
\renewcommand{\labelitemii}{$\circ$}

\newcommand{\im}{\operatorname{im}}
\newcommand{\coker}{\operatorname{coker}}

\newcommand\numberthis{\addtocounter{equation}{1}\tag{\theequation}}

% Display math
\newcommand{\ssfrac}[2]{
	\raisebox{+0.3ex}{$#1$}
	/
	\raisebox{-0.3ex}{$#2$}
}
% Inline math
\newcommand{\sfrac}[2]{
	\raisebox{+0.3ex}{\scalebox{0.9}{$#1$}}
	/
	\raisebox{-0.3ex}{\scalebox{0.9}{$#2$}}
}

 % No white line in equal arrows
\usetikzlibrary{decorations.markings}
\tikzset{double line with arrow/.style args={#1,#2}{decorate,decoration={markings,%
                    mark=at position 0 with {\coordinate (ta-base-1) at (0,1pt);
                            \coordinate (ta-base-2) at (0,-1pt);},
                    mark=at position 1 with {\draw[#1] (ta-base-1) -- (0,1pt);
                            \draw[#2] (ta-base-2) -- (0,-1pt);
                        }}}}
\tikzset{Equal/.style={-,double line with arrow={-,-}}}

% Arrow
\newcommand\Iso{\xrightarrow{
                \,\smash{\raisebox{-0.65ex}{\ensuremath{\scriptstyle\sim}}}\,}}


\newtheorem{theorem}{Theorem}[section]
\newtheorem{lemma}[theorem]{Lemma}

\theoremstyle{definition}
\newtheorem{definition}[theorem]{Definition}

\theoremstyle{definition}
\newtheorem{example}[theorem]{Example}

\theoremstyle{remark}
\newtheorem*{remark}{Remark}

\theoremstyle{definition}
\newtheorem{exercise}{Esercizio}[section]
\newtheorem*{exercise*}{Esercizio}

\title{Elementi di Topologia Algebrica - Gruppo 2\vspace{-1cm}}
\author{}

\begin{document}
\maketitle

\textbf{Esercizio 1}
\begin{itemize}
    \item[(a)] Sia $X$ un grafo finito connesso, e procediamo per induzione sul numero di archi del grafo $X$: se il grafo ha un solo arco, allora possiede uno o due vertici, e il sottografo che selezionamo come albero è rispettivamente
        \begin{tikzcd}
            \bullet
        \end{tikzcd}
        oppure
        \begin{tikzcd}
            \bullet \arrow[r, no head] & \bullet
        \end{tikzcd}.
        Supponiamo ora che $X$ abbia $n$ archi, e senza perdita di generalità supponiamo che $X$ non sia un albero. Allora, esiste un vertice $x$ interno ad un qualche arco $a$ tale che $X\setminus\{x\}$ è connesso, e notiamo che $X\setminus\{a\}$ è un grafo connesso con $n-1$ archi, in quanto possiamo rimuovere il segmento $a\simeq [0,1]$ senza perdere la connessione. Possiamo quindi applicare l'ipotesi induttiva su tale grafo con $n-1$ archi, ed avendo questo gli stessi verticidi $X$ l'albero $T$ che otteniamo è il sottografo di $X$ cercato.

    \item[(b)] Sia $T$ un albero, e procediamo ancora per induzione sul numero di archi: se $T$ ha un solo arco, allora è omeomorfo a $[0,1]$ ed è quindi contraibile.

        Altrimenti, asseriamo che $T$ contiene un vertice di grado 1: infatti, osserviamo che un albero è equivalentemente un grafo connesso minimale, e quindi il numero di archi di $T$ è uguale al numero di vertici meno uno. Ora, se tutti i vertici di $T$ avessero grado maggiore o uguale a 2, allora il numero di archi sarebbe almeno il doppio del numero di vertici, che è un assurdo.

        Sia quindi $x$ un vertice di grado 1 che si trova sull'arco $a$; allora, possiamo retrarre $a\cup \{x\}$ ad un punto, ottenendo un albero con un arco in meno. Applicando l'ipotesi induttiva, otteniamo che tale albero è contraibile, e quindi possiamo concatenare le due omotopie ottenute per ottenere una contrazione di $T$ ad un punto.

    \item[(c)] Osserviamo che $X/T$ è connesso, e quindi
        \[
            H_0(X/T)\cong H_0(X)\cong\Z
            .\]
        Inoltre, notiamo che se $(X,A)$ è una coppia con $A$ contraibile, allora la successione esatta lunga in omologia relativa ci dà, per $i>0$, isomorfismi
        \begin{equation}\label{eq:rel}
            \dots \cancel{H_{i+1}(A)} \to H_i(X) \Iso H_i(X,A)\to \cancel{H_i(A)}\to \dots
            .\end{equation}

        Ora, affermiamo che $(X,T)$ è una buona coppia: infatti, per qualsiasi arco $a$, $(X,e)$ è una buona coppia, e per la costruzione del punto precedente possiamo vedere la proiezione $X\to X/T$ come una successione di quozienti successivi di un arco alla volta. Dalla teoria otteniamo quindi
        \begin{equation}\label{eq:good}
            H_i(X,T)\cong H_i(X/T)
        \end{equation}
        per ogni $i > 0$, e perciò visto che $T$ è contraibile otteniamo
        \begin{align*}
            H_i(X) & \overset{\eqref{eq:rel}}{\cong} H_i(X,T)      \\
                   & \overset{\eqref{eq:good}}{\cong} H_i(X/T,T/T) \\
                   & = H_i(X/T,*)                                  \\
                   & \overset{\eqref{eq:rel}}{\cong} H_{i}(X/T).
        \end{align*}

        Il fatto che $X/T$ sia topologicamente omeomorfo ad un wedge di $S^1$ segue dal fatto che $T$ contiene tutti i vertici di $X$, che vengono quindi tutti identificati fra di loro. Di conseguenza, $X/T$ ha un singolo vertice, e tutti gli archi sono loop di tale vertice:    \begin{tikzcd}
            \bullet \arrow[loop, distance=2em, in=120, out=60, no head] \arrow[loop, distance=2em, in=210, out=150, no head]  \arrow[loop, distance=2em, in=300, out=240, no head] \arrow[loop, distance=2em, in=30, out=330, no head]
        \end{tikzcd}.

    \item[(d)] Calcoliamo la caratteristica di Eulero di un albero $T$ come segue: essendo $T$ un grafo, sappiamo che
        \[
            \chi(T) = \operatorname{rk}{(H_0(T))} - \operatorname{rk}{(H_1(T))},
        \]
        e dal fatto che $T$ è contraibile e connesso deduciamo $\chi(T)= 1-0 = 1$.

    \item[(e)] Innanzitutto $H_0(X/T)=\Z$ perchè $X/T$ è connesso; inoltre, visto che $X/T$ è topologicamente un wedge di $S^1$ abbiamo che

        \[
            H_1(X/T) = \pi_1(X/T)^{ab} = (\underbrace{\Z*\dots*\Z}_{r})^{ab}=\Z^r,
        \]
        dove $r$ è il numero di archi di $X/T$, che possiamo determinare come segue:
        \begin{align*}
            k & = |\{\text{archi di }X\}| - |\{\text{archi di }T\}|                       \\
              & = (|\{\text{vertici di }X\}| - \chi(X)) - (|\{\text{vertici di }T\}| - 1) \\
              & = 1-\chi(X),
        \end{align*}
        dove abbiamo usato il fatto che $X$ e $T$ hanno gli stessi vertici.

        Inoltre, la struttura cellulare di $X/T$ è data incollando gli estremi di $k$ 1-celle ad una 0-cella, e quindi visto che l'omologia cellulare coincide con l'omologia singolare concludiamo che i gruppi di omologia di $X/T$ (e quindi anche di $X$) sono
        \[
            H_0(X/T) = \Z, \quad H_1(X/T) = \Z^{1-\chi(X)}, \quad H_i(X/T) = 0 \text{ per } i>1.
        \]
\end{itemize}

\textbf{Esercizio 2}
\begin{itemize}
    \item[(a)] Vogliamo mostrare che la seguente composizione è un isomorfismo:

        \[
            \begin{tikzcd}
                H_n(D_\sigma^n,\partial D_\sigma^n) \arrow[r, "(f_\sigma)_*"] & H_n(K^{(n)},B_\sigma) \arrow[r, "(p_\sigma)_*"] & \widetilde{H}_n(S_\sigma^n),
            \end{tikzcd}
        \]
        dove per comodità abbiamo indicato con $B_\sigma$ lo spazio $K^{(n)}\setminus f_\sigma(\mathring{D_\sigma^n})$.

        La prima mappa è iniettiva per un teorema visto in classe sugli spazi ottenuti per incollamento di $n$-celle; inoltre, nello stesso teorema abbiamo che
        \[
            H_n(K^{(n)},B_\sigma) \cong \im{((f_\sigma)_*)},
        \]
        visto che $K^{(n)}$ è ottenuto da $B_\sigma$ attaccando la $n$-cella $f_\sigma(\mathring{D}_\sigma^n)$; questo mostra che $(f_\sigma)_*$ è suriettiva, ma allora è un isomorfismo.

        Inoltre, la seconda mappa è un isomorfismo perchè $(K^{(n)},B_\sigma)$ è una buona coppia (in quanto sottocomplesso di un CW complesso, come mostrato ad esempio nella Proposizione A.5 dell'Hatcher) e $(p_\sigma)_*$ è indotta dalla mappa di proiezione associata a tale coppia.

    \item[(b)] Se $\tau\neq \sigma$, allora visto che le $n$-celle sono tutte disgiunte si ha
        \[f_\tau(\mathring{D}_\sigma^n)\subset B_\sigma,\]
        ma d'altronde il bordo di $D_\sigma^n$ ha sempre questa proprietà, e quindi la mappa $p_\sigma$ manda tutto $f_\tau(\mathring{D}_\sigma^n)$ in un punto; di conseguenza, la composizione \[(p_\sigma\circ f_\tau)_* = (p_\sigma)_*\circ (f_\tau)_*\] è la mappa nulla sui gruppi omologia, quanto indotta da una mappa costante.
\end{itemize}

\textbf{Esercizio 3}
\begin{itemize}
    \item[(a)] Per un teorema visto in classe, essendo $K^{(n)}\cup Y$ ottenuto da $K^{(n-1)}\cup Y$ attaccando $n$-celle,
abbiamo che se $i\neq n$ allora
\[
    H_i(K^{(n)}\cup Y, K^{(n-1)}\cup Y) = 0.
\]
D'altronde, per $i=n$ possiamo usare il medesimo teorema ed il fatto che $H_n(D^n,\partial D^n)$ per
dedurre che
\[
    H_n(K^{(n)}\cup Y, K^{(n-1)}\cup Y) = \sum_{\sigma} \im((f_\sigma)_*) \cong \Z^{|\{\sigma\}|}.
\]
Qui, gli indici $\sigma$ variano fra le mappe di incollamento utilizzate per costruire il CW complesso $K^{(n)}\cup Y$ a partire da $K^{(n-1)}\cup Y$. Per definizione di sottocomplesso, queste sono tante quante le $n$-celle di
$K^{(n)}$ che non intersecano $Y$.
\item[(b)] Partiamo con il caso $i<n$: per un teorema visto in classe, sappiamo che l'inclusione dell'$n$-scheletro di un CW complesso in esso induce un isomorfismo in
omologia per $i<n$. Utilizzando questo fatto e la fattorizzazione associata alle inclusioni successive \[
    K^{(n)}\hookrightarrow K^{(n)}\cup Y \overset{j}{\hookrightarrow} X
,\]
otteniamo il diagramma
\[
    \begin{tikzcd}
        H_i(K^{(n)}\cup Y) \arrow[r,"j_*"]  & H_i(X) \\
        H_i(K^{(n)}) \arrow[u, "\cong"] \arrow[ru, "\cong"]
    \end{tikzcd}
\]
Questo implica che anche $j_*$ è un isomorfismo per $i<n$. Per concludere, usiamo il morfismo (indotto dalle inclusioni) fra le successioni esatte lunghe in omologia relativa di $X$ associate alle coppie
$(X,K^{(n)}\cup Y)$ ed $(X,X)$:
\[
    \begin{tikzcd}[column sep=tiny]
        \dots \arrow[r] & H_i(K^{(n)}\cup Y) \arrow[d, "\cong"] \arrow[r] & H_i(X) \arrow[d, "\cong"] \arrow[r] & H_i(X,K^{(n)}\cup Y) \arrow[d] \arrow[r] & H_{i-1}(K^{(n)}\cup
        Y) \arrow[d, "\cong"] \arrow[r] & H_{i-1}(X) \arrow[d, "\cong"] \arrow[r] & \dots \\
        \dots \arrow[r] & H_n(X) \arrow[r] & H_n(X) \arrow[r] & \cancel{H_i(X,X)} \arrow[r] & H_{n-1}(X) \arrow[r] & H_{n-1}(X) \arrow[r] & \dots
    \end{tikzcd}
\]
Da cui, per il lemma dei cinque, la mappa verticale centrale è un isomorfismo e quindi $H_i(X,K^{(n)}\cup Y) = 0$ per $i<n$.

Per $i=n$, invece, consideriamo la successione esatta lunga della tripla \[(X,K^{(n+1)}\cup Y,K^{(n)}\cup Y):\]
    \begin{align*}
        & \dots \to  H_n(K^{(n+1)}\cup Y,K^{(n)}\cup Y) \to H_n(X,K^{(n)}\cup Y) \to  \\
         \to & H_n(X,K^{(n+1)}\cup Y)\to  H_{n-1}(K^{(n+1)}\cup Y,K^{(n)}\cup Y) \to  \dots
    \end{align*}
    I due estremi sono nulli per il punto $(a)$ dell'esercizio, mentre il terzo termine è nullo perchè per $i<n+1$ si ha $H_i(X,K^{(n+1)}\cup Y)=0$ per quanto detto sopra; questo
    mostra la tesi per $i=n$.

    Infine, se $i<n$ consideriamo la successione esatta lunga della tripla \[(X,K^{(n)}\cup Y,Y):\]
    \[
        \begin{tikzcd}[column sep=tiny]
            \dots \arrow[r] & H_{i+1}(X,    K^{(n)}\cup Y) \arrow[r] & H_{i}(K^{(n)}\cup Y,Y) \arrow[r] & H_{i}(X,Y) \arrow[r] & H_{i}(X, K^{(n)}\cup Y) \arrow[r] & \dots
        \end{tikzcd}
    \]
    Visto che $i\leq n+1$, i due estremi sono nulli per quanto appena dimostrato, e quindi la mappa centrale è un isomorfismo per esattezza.
\item[(c)] Per definire il complesso $C_*(K,Y)$, consideriamo le successioni esatte lunghe associate alle coppie $(K^{(n)}\cup Y, K^{(n-1)}\cup Y)$ al variare di $n$: più
    precisamente, posto $$C_n(K,Y)\coloneqq H_n(K^{(n)}\cup Y, K^{(n-1)}\cup Y),$$ definiamo i differenziali come la composizione
    \[
        \begin{tikzcd}[column sep=small]
            H_n(K^{(n)}\cup Y, K^{(n-1)}\cup Y) \arrow[r] & H_{n-1}(K^{(n-1)}\cup Y) \arrow[r] & H_{n-1}(K^{(n-1)}\cup Y, K^{(n-2)}\cup Y)
        \end{tikzcd}
    \]
    dove la prima mappa è quella di connessione relativa alla coppia $$(K^{(n)}\cup Y, K^{(n-1)}\cup Y),$$ mentre la seconda è quella indotta dalla proiezione relativa alla coppia
    $$(K^{(n-1)}\cup Y, K^{(n-2)}\cup Y).$$

    Per costruire l'isomorfismo cercato fra l'omologia di tale complesso e l'omologia di $X$ relativa ad $Y$, utilizzando le successioni delle triple $(K^{n+1}\cup Y, K^{(n)}\cup
    Y, K^{(n-1)}\cup Y)$ e $(K^{(n)}\cup Y, K^{(n-1)}\cup Y, K^{(n-2)}\cup Y)$ si può usare l'esattezza ed escissione per mostrare che $$H_n(C_*(K,Y))\cong \ssfrac{H_n(K^{(n)},
    K^{(n-1)}\cup Y)}{H_n(X, K^{(n-1)}\cup Y)\cong H_n(X,Y)},$$ utilizzando le successione delle rispettive coppie per ottenere l'ultimo isomorfismo.
\end{itemize}
\textbf{Esercizio 4}
\begin{itemize}
    \item[(a)] Costruiamo una retrazione di $M_f$ sulla copia di Y contenuta in $M_f$ a tempo $1$: consideriamo infatti
        \begin{align*}
            H & : M_f \times [0,1]  \to M_f          \\
              & \begin{cases}
                    ((x,s),t) & \mapsto (x, s(1-t) + t), \\
                    (y,t)     & \mapsto y.
                \end{cases}
        \end{align*}
        Osserviamo che $H$ è ben definita e continua; d'altronde, $H_t$ ristretta ad Y è l'identità per ogni $t$, $H_0$ è l'identità di $M_f$ ed $H_1(M_f)=Y$. Quindi, $H$ è una retrazione per deformazione di $M_f$ su $Y$, e ponendo $p=H_1$ otteniamo l'equivalenza omotopica cercata:
        \[
            p\circ \iota (x) = H_1\circ \iota (x) = H_1(x,0) = (x,1) = f(x)
        \]
        per definizione di mapping cone.

    \item[(b)] Consideriamo la successione esatta lunga in omologia relativa associata alla coppia $(M_f, X)$:
        \begin{equation}\label{eq:couple}
            \begin{tikzcd}
                \dots \arrow[r] & \widetilde{H}_i(X) \arrow[r] & \widetilde{H}_i(M_f) \arrow[r] & H_i(M_f, X) \arrow[r] & \widetilde{H}_{i-1}(X) \arrow[r] & \dots,
            \end{tikzcd}
        \end{equation}
        dove abbiamo usato il fatto che l'omologia della coppia coincide l'omologia ridotta.

        Visto che $Y$ è un retratto per deformazione di $M_f$ e che $(M_f, X)$ è una buona coppia (ad esempio perché $X\times [0,\frac{1}{2})$ si retrae per deformazione su $X$),
        osservando che
        \[
            \ssfrac{M_f}{X} \cong C_f
        \]
        deduciamo gli isomorfismi
        \begin{align*}
            \widetilde{H}_i(M_f) & \cong  \widetilde{H}_i(Y)   \\
            H_i(M_f, X)          & \cong \widetilde{H}_i(C_f).
        \end{align*}
        Sostituendoli nella \eqref{eq:couple}, otteniamo quindi la successione esatta desiderata:
        \[
            \begin{tikzcd}
                \dots \arrow[r] & \widetilde{H}_i(X) \arrow[r] & \widetilde{H}_i(Y) \arrow[r] & \widetilde{H}_i(C_f) \arrow[r] & \widetilde{H}_{i-1}(X) \arrow[r] & \dots
            \end{tikzcd}
        \]
\end{itemize}

\end{document}
