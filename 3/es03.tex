\documentclass[a4paper]{article}

\usepackage[T1]{fontenc}
% many useful symbols
\usepackage{textcomp}
\usepackage[italian]{babel}
\usepackage{hyperref}
\usepackage{amsmath, amssymb, amsthm}
\usepackage{mathtools}
% for \lightning
\usepackage{stmaryrd}
\usepackage{geometry}
\usepackage{tikz-cd}
\usepackage{cancel}

% Remove indentation globally
\setlength{\parindent}{0pt}
% Have blank lines between paragraphs
\usepackage[parfill]{parskip}

\hypersetup{
    colorlinks = true, % links instead of boxes
    urlcolor   = cyan, % external hyperlinks
    linkcolor  = blue, % internal links
    citecolor  = cyan   % citations
}

\newcommand{\R}{\mathbb{R}}
\newcommand{\C}{\mathbb{C}}
\newcommand{\Q}{\mathbb{Q}}
\newcommand{\N}{\mathbb{N}}
\newcommand{\A}{\mathbb{A}}
\newcommand{\Z}{\mathbb{Z}}

% use bullets for items
\renewcommand{\labelitemii}{$\circ$}

\newcommand{\im}{\operatorname{im}}
\newcommand{\coker}{\operatorname{coker}}

\newcommand\numberthis{\addtocounter{equation}{1}\tag{\theequation}}

% Display math
\newcommand{\ssfrac}[2]{
	\raisebox{+0.3ex}{$#1$}
	/
	\raisebox{-0.3ex}{$#2$}
}
% Inline math
\newcommand{\sfrac}[2]{
	\raisebox{+0.3ex}{\scalebox{0.9}{$#1$}}
	/
	\raisebox{-0.3ex}{\scalebox{0.9}{$#2$}}
}

 % No white line in equal arrows
\usetikzlibrary{decorations.markings}
\tikzset{double line with arrow/.style args={#1,#2}{decorate,decoration={markings,%
                    mark=at position 0 with {\coordinate (ta-base-1) at (0,1pt);
                            \coordinate (ta-base-2) at (0,-1pt);},
                    mark=at position 1 with {\draw[#1] (ta-base-1) -- (0,1pt);
                            \draw[#2] (ta-base-2) -- (0,-1pt);
                        }}}}
\tikzset{Equal/.style={-,double line with arrow={-,-}}}

% Arrow
\newcommand\Iso{\xrightarrow{
                \,\smash{\raisebox{-0.65ex}{\ensuremath{\scriptstyle\sim}}}\,}}


\newtheorem{theorem}{Theorem}[section]
\newtheorem{lemma}[theorem]{Lemma}

\theoremstyle{definition}
\newtheorem{definition}[theorem]{Definition}

\theoremstyle{definition}
\newtheorem{example}[theorem]{Example}

\theoremstyle{remark}
\newtheorem*{remark}{Remark}

\theoremstyle{definition}
\newtheorem{exercise}{Esercizio}[section]
\newtheorem*{exercise*}{Esercizio}

\title{Elementi di Topologia Algebrica - Gruppo 3\vspace{-1cm}}
\author{}

\begin{document}
\maketitle

\textbf{Esercizio 1}

Fissato $n$, vogliamo mostrare che il seguente diagramma commuta:
\[
    \begin{tikzcd}
        C_n(X) \arrow[r, "\delta_n"] \arrow[d, "D"] & C_{n-1}(X) \arrow[d, "D"] \\
        C_n(X)\otimes C_n(X) \arrow[r, "\delta_n^\prime"] & C_{n-1}(X)\otimes C_{n-1}(X)
    \end{tikzcd}
\]
Calcoliamo entrambe le composizioni: per $\sigma\in C_n(X)$, abbiamo
\begin{align*}
    D\circ\delta_n(\sigma) &= D\left(\sum_{i=0}^n (-1)^i\sigma\circ d_i\right)\\
                           &= \sum_{i=0}^n (-1)^i\left(\sum_{p+q=n-1} (\sigma\circ d_i)_p^1\otimes(\sigma\circ d_i)_q^2\right)\\
                           &= \sum_{i=0}^{p} (-1)^i\left(\sum_{p+q=n-1} (\sigma\circ d_i)_p^1\otimes(\sigma\circ d_i)_q^2\right) +  \sum_{i=p+1}^n (-1)^i\left(\sum_{p+q=n-1} (\sigma\circ d_i)_p^1\otimes(\sigma\circ d_i)_q^2\right).
\end{align*}
Osserviamo inoltre che valgono le seguenti relazioni, che si ottengono esplicitando le mappe $a_p$, $b_q$ in termini delle mappe $d_i$:
\[
\begin{cases}
    (\sigma\circ d_i)_p^1 = \sigma_{p+1}^1\circ d_i & \text{se } i\leq p,\\
    (\sigma\circ d_i)_p^1 = \sigma_p^1 & \text{se } i>p,\\
    (\sigma\circ d_i)_q^2 =  \sigma_q^2 & \text{se } i\leq p,\\
    (\sigma\circ d_i)_q^2 = \sigma_{q+1}^2\circ d_{i-p} & \text{se } i > p.
\end{cases}
\]
Effettuando tali sostituzioni e riscrivendo $q$ in termini di $n$ e $p$, otteniamo:
\begin{align*}
    &D   \circ \delta_n(\sigma) = \\
                               &\quad= \sum_{p=0}^{n-1} \left(\sum_{i=0}^{p} (-1)^i (\sigma_{p+1}^1\circ d_i)\otimes\sigma_{n-p-1}^2 +  \sum_{i=p+1}^n (-1)^i \sigma_p^1 \otimes (\sigma_{n-p}^2\circ {d_{i-p}} )\right).
\end{align*}

Esplicitiamo ora l'altra composizione:
\begin{align*}
    \delta_n^\prime\circ D(\sigma) &= \delta_n^\prime\left(\sum_{p+q=n} \sigma_p^1\otimes\sigma_q^2\right)\\
                                   &= \sum_{p+q=n} \left( \left(\sum_{i=0}^p (-1)^i(\sigma_p^1\circ d_i)\right)\otimes(\sigma_q^2) + (-1)^p (\sigma_p^1)\otimes \left(\sum_{j=0}^q
                                       (-1)^j(\sigma_q^2\circ d_j)\right) \right)\\
                                   &= \sum_{p=0}^n \left( \left(\sum_{i=0}^p (-1)^i(\sigma_p^1\circ d_i)\right)\otimes(\sigma_{n-p}^2) +  \left(\sum_{j=0}^{n-p}
                                           (-1)^{j+p}\sigma_p^1\otimes(\sigma_{n-p}^2\circ d_j)\right) \right).
\end{align*}
Osserviamo ora che nell'espressione qua sopra, i termini con $i=p$ e $j=0$ si cancellano, in quanto otteniamo una somma telescopica
\[
    \sum_{p=0}^n\left((-1)^p(\sigma_1^p\circ d_p)\otimes\sigma^2_{n-p} +(-1)^p\sigma_1^p \otimes (\sigma^2_{n-p}\circ d_0) \right)=
    \sum_{p=0}^n\left((-1)^p \sigma^{p-1}_1 \otimes \sigma^2_{n-p} + (-1)^p \sigma_1^p \otimes \sigma^2_{n-p-1}\right)=0
.\]
In conclusione, per entrambe le composizioni otteniamo lo stesso numero $n(n+1)$ di termini, e permutando gli indici delle sommatorie otteniamo che i due risultati coincidono.

\textbf{Esercizio 2}
Siano $\pi_X:X\times Y\to X$ e $\pi_Y:X\times Y\to Y$ le proiezioni canoniche, che inducono mappe $\pi_X^*:C^*(X)\to C^*(X\times Y)$ e $\pi_Y^*:C^*(Y)\to C^*(X\times Y)$ sulle
cocatene. Mostreremo che l'isomorfismo $\alpha$ ammette la scrittura esplicita
\begin{equation}\label{eq:alpha}
    \alpha(a\otimes b) = \pi_X^*(a)\smile\pi_Y^*(b).
\end{equation}
Per farlo, consideriamo la mappa
\[
        \beta: C^*(X)\otimes C^*(Y) \to  C^*(X\times Y)
\]
che manda $p\otimes q\in C^i(X)\otimes C^j(Y)$ in $\pi_X^*(p)\smile\pi_Y^*(q)\in C^{i+j}(X\times Y)$; questa definisce un morfismo di complessi per la naturalità delle proiezioni e
del prodotto cup, che sono compatibili con le mappe di bordo. Inoltre, $\beta$ coincide con la mappa $EZ$ di Eilenberg-Zilber in grado $0$, in quanto quest'ultima è data da $$p\otimes
q\mapsto (\sigma\to p(\pi_X(\sigma))q(\pi_Y(\sigma)))$$ al variare di $\sigma$ in $C_0(X\times Y)$, e $\beta$ dà lo stesso risultato in grado $0$ per definizione di prodotto cup. Dunque, per
il teorema dei funtori liberi ed aciclici $EZ$ è omotopa a $\beta$, e quindi inducono la stessa mappa in coomologia. In altre parole, passando per la formula di Künneth otteniamo
che $\beta$ induce un isomorfismo
\[
    \bigoplus_{p+q=n} H^p(X)\otimes H^q(Y) \Iso H^n(X\times Y)
\]
che coincide con $\alpha$ per costruzione.
Per concludere, osserviamo che vale la seguente catena di uguaglianze (dove usiamo la scrittura \eqref{eq:alpha} per $\alpha$):
\begin{align*}
    \alpha(a_i\otimes a_j)\smile\alpha(b_p \otimes b_q) &= (\pi_X^*(a_i)\smile\pi_Y^*(a_j))\smile(\pi_X^*(b_p)\smile\pi_Y^*(b_q))\\
                                                        &= \pi_X^*(a_i)\smile(\pi_Y^*(a_j)\smile\pi_X^*(b_p))\smile\pi_Y^*(b_q) && \text{(associatività di $\smile$)}\\
                                                        &= (-1)^{(jp)}\pi_X^*(a_i)\smile  \pi_X^*(b_p)\smile\pi_Y^*(a_j)\smile\pi_Y^*(b_q) && \text{(anti-commutatività di $\smile$)}\\
                                                        &= (-1)^{jp}\pi_X^*(a_i\smile b_p)\smile\pi_Y^*(a_j\smile b_q) && (\ast)\\
                                                        &= \alpha((a_i\smile b_p)\otimes (a_j\smile b_q)),
\end{align*}
come volevamo dimostrare. Il passaggio $(\ast)$ è una verifica immediata, che riportiamo nel caso di $\pi_X^*$: se $a,b$ sono cocicli in gradi $i,j$ rispettivamente, allora per
ogni $\sigma\in C_{i+j}(X\times Y)$ abbiamo
\[
    \pi_X^*(a\smile b)(\sigma) = (-1)^{ij} a(\pi_X\circ \sigma)b(\pi_Y\circ \sigma) = \pi_X^*(a)\smile\pi_X^*(b)
.\]


\textbf{Esercizio 3}
Dimostriamo innanzitutto che i gruppi $H^i(T^n)$ sono isomorfi a $\Z^{\binom{n}{i}}$ per induzione su $n$: per $n=1$ abbiamo $T^1=S^1$, e sappiamo che $H^0(S^1)=H^1(S^1)=\Z$ ed
$H^i(S^1)=0$ per $i>1$; per il passo induttivo, notiamo che $T^n=T^{n-1}\times S^1$ è prodotto di spazi topologici con coomologia libera, e quindi per la formula di Künneth
otteniamo
\[
    H^i(T^n) \simeq H^i(T^{n-1})\otimes H^0(S^1) \oplus H^{i-1}(T^{n-1})\otimes H^1(S^1) \simeq \Z^{\binom{n-1}{i}}\oplus\Z^{\binom{n-1}{i-1}} = \Z^{\binom{n}{i}}
.\]
Per quanto riguarda la struttura moltiplicativa, mostriamo per induzione su $n$ che $H^*(T^n)$ è isomorfo alla $\Z$-algebra esterna $\Lambda(x_1,\ldots,x_n)$ su $n$ generatori dove
gli $x_i$ sono generatori dell'$H^1$ dell'$i$-esima copia di $S^1$, il cui prodotto è $T^n$: per
$n=1$, abbiamo $$H^*(S^1)=\Z \oplus x\Z$$ con $x$ generatore di $H^1(S^1)$, e vale $x^2=0$ visto che $x^2=x\smile x \in H^2(S^1)=0$; per il passo induttivo, possiamo usare la
relazione dell'esercizio precedente, per la quale l'isomorfismo $\alpha$ di Künneth risulta essere un isomorfismo di algebre graduate, e quindi
\[
    H^*(T^n)\simeq H^*(T^{n-1})\otimes H^*(S^1)
\]
come algebre graduate.
Per ipotesi induttiva, $H^*(T^{n-1})$ è isomorfo a $\Lambda(x_1,\ldots,x_{n-1})$, ed $H^*(S^1)$ è isomorfo a $\Lambda(x_n)$ con $x_n$ generatore di $H^1(S^1)$; perciò
\[
    H^*(T^n)\simeq \Lambda(x_1,\ldots,x_{n-1})\otimes\Lambda(x_n)\simeq \Lambda(x_1,\ldots,x_{n-1},x_n)
,\]
che conclude il passo induttivo.

\textbf{Esercizio 4}

Notiamo preliminarmente che essendo $X$ una varietà connessa, è anche connessa per archi. Inoltre, essendo $X$ \textit{non} compatta, abbiamo che $X\setminus K$ è non vuoto per ogni $K\subset X$ compatto.

Per definizione di coomologia a supporto compatto, preso un $0$-cociclo
\[
    \alpha\in H^0_c(X)=\ker{(\delta^0:\,C^0_c(X)\to C^1_c(X))}
,\]
questo vive in $C^0(X,X\setminus K )=\ker(C^0(X)\to C^0(X\setminus K))$ per un qualche compatto $K\subset X$. Se fissiamo $y\in X\setminus K$, per ogni $x\in X$ esiste un cammino $\gamma$ che congiunge
$x$ ad $y$.

Quindi,
\[
    \delta^0(\alpha)=0 \implies 0=\delta^0(\alpha)(\gamma)=\alpha(y)-\alpha(x)=0,
\] ma visto che $y\in X\setminus K$, abbiamo $\alpha(y)=0$ e quindi $\alpha(x)=0$ per ogni $x\in X$ per l'arbitrarietà di $x$; concludiamo perciò $\alpha=0$ e dunque $H^0_c(X)=0$.
\end{document}
